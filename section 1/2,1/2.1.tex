\documentclass[12pt]{article}
\usepackage{lingmacros}
\usepackage{tree-dvips}
\usepackage[utf8]{inputenc}
\usepackage[russian]{babel}
\usepackage{amsmath,amssymb}
\usepackage{multirow}
\usepackage{hyperref}
\usepackage{caption}
\usepackage{tabularx}



\begin{document}

\section*{Задача 2.1}
\subsection*{(1)}
\paragraph*{}
Ще напрваим индукция по дефиницията на частичната субституция.
\begin{enumerate}
	\item $M \equiv x$. В този случай $M[x \rightsquigarrow N] \equiv N$ и $M[x \hookrightarrow N] \equiv N$.
	\item $M \equiv y$, $y \not\equiv x$. В този случай $M[x \rightsquigarrow N] \equiv y$ и $M[x \hookrightarrow N] \equiv y$.
	\item $M \equiv M_1M_2$. Тогава $M[x \rightsquigarrow N] = (M_1[x \rightsquigarrow N])(M_2[x \rightsquigarrow] N)$ и $M[x \hookrightarrow N] = (M_1[x \hookrightarrow N])(M_2[x \hookrightarrow N])$. От ИП $(M_1[x \rightsquigarrow N]) \equiv (M_1[x \hookrightarrow N])$ и $(M_2[x \rightsquigarrow N]) \equiv (M_2[x \hookrightarrow N])$. Също от ИП знаем, че двете частични субституции са дефинирани. Така получаваме, че $M[x \rightsquigarrow N] \equiv M[x \hookrightarrow N]$, а също сме и сигурни, че $M[x \hookrightarrow N]$ е дефинирано.
	\item $M = \lambda_x P$. Тогава $M[x \rightsquigarrow N] \equiv \lambda_x P$ и $M[x \hookrightarrow N] \equiv \lambda_x P$, така получаваме, че $M[x \rightsquigarrow N] \equiv M[x \hookrightarrow N]$.
	\item $M = \lambda_y P$ за $y \not\equiv x$. Ясно е, че ако частичната субституция е дефинирана в този случай, то резултатите от двете субституции ще съпвадат. Това, което трябва да се покаже е, че частичната субституция е дефинирана в този случай. По-конкретно трябва да покажем, че $x \not\in FV(P)$ или $y \not\in FV(N)$. Ще използваме допускането от задачата, че $FV(N) \cap BV(M) = \{ \}$. 
	\paragraph*{}
	От дефиницията на $BV$ може да се види, че $y \in BV(M)$. Понеже $FV(N) \cap BV(M) = \{ \}$, то излиза, че $y \not\in FV(N)$. Това показва, че частичната субституция е дефинирана.
	\paragraph*{}
	Също така от ИП имаме, че $P[x \hookrightarrow N] \equiv P[x \rightsquigarrow N]$, от което и получаваме, че $М[x \hookrightarrow N] \equiv М[x \rightsquigarrow N]$

\subsection*{(2)}
\paragraph*{}
Да разгледаме терма $M = \lambda_y \lambda_x y$. Искаме да направим субституция като заместим $x$ с $y$. Забелязваме, че частичната субституция $M[x \hookrightarrow y]$ е дефинирана и дава $\lambda_y \lambda_x y$. Но наивната субституция не е коректна, понеже $BV(M) = \{ x, y \}$, а $FV(N) = \{ y \}$ (тук $N$ e $y$) и съответно нямат празно сечение.

\end{enumerate}

\end{document}