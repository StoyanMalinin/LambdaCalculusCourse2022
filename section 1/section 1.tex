\documentclass[12pt]{article}
\usepackage{lingmacros}
\usepackage{tree-dvips}
\usepackage[utf8]{inputenc}
\usepackage[russian]{babel}
\usepackage{amsmath,amssymb}
\usepackage{multirow}
\usepackage{hyperref}
\usepackage{caption}
\usepackage{tabularx}

\begin{document}

\newcommand\alphaeq{\mathrel{\stackrel{\makebox[0pt]{\mbox{\normalfont\tiny $\alpha$}}}{=}}}
\newcommand\betaeq{\mathrel{\stackrel{\makebox[0pt]{\mbox{\normalfont\tiny $\beta$}}}{=}}}
\newcommand\betaetaeq{\mathrel{\stackrel{\makebox[0pt]{\mbox{\normalfont\tiny $\beta\eta$}}}{=}}}


\section*{Задача 2.3}
\begin{itemize}
		\item $(A, A) \in \alphaeq$, $A \in \Lambda$
		\item ако $(A_1, A_2) \in \alphaeq$, то $(A_2, A_1) \in \alphaeq$, $A_1, A_2 \in \Lambda$
		\item ако $(A_1, A_2) \in \alphaeq$ и $(A_2, A_3) \in \alphaeq$, то $(A_1, A_3) \in \alphaeq$, $A_1, A_2, A_3 \in \Lambda$
		\item ако $(A_1, B_1) \in \alphaeq$ и $(A_2, B_2) \in \alphaeq$, то $(A_1 B_1, A_2 B_2) \in \alphaeq$, $A_1, A_2, B_1, B_2 \in \Lambda$
		\item Нека имаме $M = \lambda_x A$, $N = \lambda_y B$, $A, B \in \Lambda$ и $x, y \in V$. Ако $(A, B[y \rightsquigarrow x]) \in \alphaeq$ и $y \not\in FV(A) \cup BV(A)$, то $(M, N) \in \alphaeq$ (тук $\rightsquigarrow$ е наивна субституция).
\end{itemize}

\section*{Задача 2.1}
\subsection*{(1)}
\paragraph*{}
Ще напрваим индукция по дефиницията на частичната субституция.
\begin{enumerate}
	\item $M \equiv x$. В този случай $M[x \rightsquigarrow N] \equiv N$ и $M[x \hookrightarrow N] \equiv N$.
	\item $M \equiv y$, $y \not\equiv x$. В този случай $M[x \rightsquigarrow N] \equiv y$ и $M[x \hookrightarrow N] \equiv y$.
	\item $M \equiv M_1M_2$. Тогава $M[x \rightsquigarrow N] = (M_1[x \rightsquigarrow N])(M_2[x \rightsquigarrow] N)$ и $M[x \hookrightarrow N] = (M_1[x \hookrightarrow N])(M_2[x \hookrightarrow N])$. От ИП $(M_1[x \rightsquigarrow N]) \equiv (M_1[x \hookrightarrow N])$ и $(M_2[x \rightsquigarrow N]) \equiv (M_2[x \hookrightarrow N])$. Също от ИП знаем, че двете частични субституции са дефинирани. Така получаваме, че $M[x \rightsquigarrow N] \equiv M[x \hookrightarrow N]$, а също сме и сигурни, че $M[x \hookrightarrow N]$ е дефинирано.
	\item $M = \lambda_x P$. Тогава $M[x \rightsquigarrow N] \equiv \lambda_x P$ и $M[x \hookrightarrow N] \equiv \lambda_x P$, така получаваме, че $M[x \rightsquigarrow N] \equiv M[x \hookrightarrow N]$.
	\item $M = \lambda_y P$ за $y \not\equiv x$. Ясно е, че ако частичната субституция е дефинирана в този случай, то резултатите от двете субституции ще съпвадат. Това, което трябва да се покаже е, че частичната субституция е дефинирана в този случай. По-конкретно трябва да покажем, че $x \not\in FV(P)$ или $y \not\in FV(N)$. Ще използваме допускането от задачата, че $FV(N) \cap BV(M) = \{ \}$. 
	\paragraph*{}
	От дефиницията на $BV$ може да се види, че $y \in BV(M)$. Понеже $FV(N) \cap BV(M) = \{ \}$, то излиза, че $y \not\in FV(N)$. Това показва, че частичната субституция е дефинирана.
	\paragraph*{}
	Също така от ИП имаме, че $P[x \hookrightarrow N] \equiv P[x \rightsquigarrow N]$, от което и получаваме, че $М[x \hookrightarrow N] \equiv М[x \rightsquigarrow N]$

\subsection*{(2)}
\paragraph*{}
Да разгледаме терма $M = \lambda_y \lambda_x y$. Искаме да направим субституция като заместим $x$ с $y$. Забелязваме, че частичната субституция $M[x \hookrightarrow y]$ е дефинирана и дава $\lambda_y \lambda_x y$. Но наивната субституция не е коректна, понеже $BV(M) = \{ x, y \}$, а $FV(N) = \{ y \}$ (тук $N$ e $y$) и съответно нямат празно сечение.

\end{enumerate}

\section*{Задача 2.23}
\paragraph{}
Нека дефинираме $c_i = \lambda_n n c_s c_0$
\begin{enumerate}
	\item Да се докаже, че за произволно $n \in \mathbb{N}$ e изпълнено $c_i c_n \betaeq c_n$.
	\item Вярно ли е, че $c_i \betaetaeq I$?
\end{enumerate}

\subsection*{1}
Ще докажем твърдението с индукция $n \in \mathbb{N}$. Преди това ще забележим, че $c_i c_n \betaeq c_n c_s c_0 = (\lambda_f \lambda_x f^n x) c_s c_0 \betaeq c_s^n c_0 $.
\subsubsection*{База}
\paragraph*{}
При $n=0$ имаме, $c_i c_0 \betaeq c_0 c_s c_0 = (\lambda_f \lambda_x x) c_s c_0 \betaeq c_0$, с което базата е доказана.

\subsubsection*{Индуктивна стъпка}
\paragraph*{}
Нека се опитаме да докажем твърдението за $n+1$. $c_i c_{n+1} \betaeq c_{n+1} c_s c_0 = (\lambda_f \lambda_x f^{n+1} x) c_s c_0 \betaeq c_s^{n+1} c_0$. Използваме дефиницията за $n$-кратна композиция на функция и записваме, че $c_s^{n+1} c_0 = c_s (c_s^n c_0)$. От наблюдението горе се сещаме, че $c_s (c_s^n c_0) \betaeq c_s (c_i c_n)$. От ИП можем да запишем, че $c_s (c_i c_n) \betaeq c_s c_n$. Сега от свойствата на $c_s$ получаваме, че $c_s c_n \betaeq c_{n+1}$. По този начин индуктивната стъпка е завършена.

\subsection*{2}
\paragraph*{}
Нека разгледаме $K = \lambda_x \lambda_y x$. Нека приложим $K$ на $c_i$ и на $I$. $I K \betaeq K = A_1$. От друга страна $c_i K \betaeq K c_s c_0 \betaeq c_s = A_2$. Използваме дефиницията за $c_s$ от лекции $c_s = \lambda_n \lambda_f \lambda_x f (n f x)$. Вижда се, че $A_1$ и $A_2$ са няма как да са бета-еквивалентни, понеже и двете са в бета-нормална форма, но не са равни. 

\section*{Задача 2.29}
\begin{itemize}
	\item $[]$ - $\lambda_f \lambda_e e$.
	\item $[x_1]$ - $\lambda_f \lambda_e ((fe)x_1)$	
	\item $[x_1, x_2]$ - $\lambda_f \lambda_e (f((fe)x_1))x_2$
	\item $[x_1, x_2, ...., x_{n+1}]$ - $\lambda_f \lambda_e (f((l_{[x_1, x_2, ...., x_{n}]} f) e)) x_{n+1}$, където $l_{[x_1, x_2, ...., x_{n}]}$ е списъка $[x_1, x_2, ...., x_{n}]$. 	
\end{itemize}

Желаните функции са реализирани във файла lists.scm. Append се казва pushBack. Функцията member е написана да сравнява само числа, за да се тества по лесно.

\section*{Задача 2.38}
\paragraph*{}
Ще направим индукция по дефиницията на $M$.

\subsection*{Случай 1 - $M \equiv y$, у е променлива}
\subsubsection*{Случай 1.1 - $y \equiv x$}
Тогава $M[x \longmapsto N] \equiv N$ и $M[x \longmapsto N'] \equiv N'$. От условието е ясно, че $(M[x \longmapsto N], M[x \longmapsto N']) \in D^{\lambda, R, T}$ по условие.
\subsubsection*{Случай 1.2 - $y \not\equiv x$}
Тогава $M[x \longmapsto N] \equiv y$ и $M[x \longmapsto N'] \equiv y$. От условието е ясно, че $(M[x \longmapsto N], M[x \longmapsto N']) \in D^{\lambda, R, T}$ заради рефлексивността.

\subsection*{Случай 2 - $M \equiv M_1 M_2$}
От ИП имаме, че $(M_1[x \longmapsto N], M_1[x \longmapsto N']) \in D^{\lambda, R, T}$ и $(M_2[x \longmapsto N], M_2[x \longmapsto N']) \in D^{\lambda, R, T}$. Тъй като $M[x \longmapsto N] \equiv (M_1[x \longmapsto N])(M_2[x \longmapsto N])$ и $M[x \longmapsto N'] \equiv (M_1[x \longmapsto N'])(M_2[x \longmapsto N'])$, то от ламбда затварянето и транзитивността следва, че $(M[x \longmapsto N], M[x \longmapsto N']) \in D^{\lambda, R, T}$.

\subsection*{Случай 3 - $M \equiv \lambda_y P$}
\subsubsection*{Случай 3.1 - $y \equiv x$}
$(\lambda_x P)[x \longmapsto N] \equiv \lambda_x P$ и $(\lambda_x P)[x \longmapsto N'] \equiv \lambda_x P$. От рефликсвоност е вярно, че $(M[x \longmapsto N], M[x \longmapsto N']) \in D^{\lambda, R, T}$.
\subsection*{Случай 3.2 - $y \not\equiv x$}
$(\lambda_y P)[x \longmapsto N] \equiv \lambda_y (P [x \longmapsto N])$ и $(\lambda_y P)[x \longmapsto N'] \equiv \lambda_y (P [x \longmapsto N'])$. От ИП имаме, че $(P [x \longmapsto N], P [x \longmapsto N']) \in D^{\lambda, R, T}$. Так от ламбда затварянето имаме, че $(M[x \longmapsto N], M[x \longmapsto N']) \in D^{\lambda, R, T}$.

\end{document}