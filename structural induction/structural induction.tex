\documentclass[12pt]{article}
\usepackage{lingmacros}
\usepackage{tree-dvips}
\usepackage[utf8]{inputenc}
\usepackage[russian]{babel}
\usepackage{amsmath,amssymb}
\usepackage{multirow}
\usepackage{hyperref}
\usepackage{caption}
\usepackage{tabularx}

\begin{document}

\section*{Задача 1.6}
\paragraph*{}

\subsection*{подзадача 1}
\paragraph*{}
Трябва да докажем, че $n+s(m)=s(n+m)$. Ще използваме структурна индукция.
\subsubsection*{случай 1: $n = o$}
\paragraph*{}
Тогава трябва да покажем, че $o + s(m) = s(o+m)$. Използвайки базата на дефиницията на събирането, това свойство излиза тривиално.
\subsubsection*{случай 2: $n \neq s(x)$, $x \in N$}
\paragraph*{}
Имаме, че $n+s(m) = s(x)+s(m)$. По дефиниция имаме, че $s(x)+s(m) = s(x + s(m))$. От ИП имаме, че $x+s(m) = s(x+m)$. От дефиницията имаме, че $s(x+m) = s(x) + m$. Така получаваме, че $s(x+s(m)) = s(s(x)+m) = s(n+m)$. 

\subsection*{подзадача 2}

\paragraph*{}
Преди това ще докажем едно помощно твърдение: $a+o = a$.
\subsubsection*{случай 1: $a = o$}
$a + o = o + o = o = a$.
\subsubsection*{случай 1: $a = s(x)$, $x \in N$}
$a + o = s(x) + o = s(x + o) = s(x) = a$ от ИП.

\paragraph*{}
Трябва да покажем, че така дефинираното събиране е комутативно, тоест $n+m=m+n$.
\subsubsection*{случай 1: $n=o$}
\paragraph*{}
Имаме, че $n+m = o+m = m = m + o = m + n$.
\subsubsection*{случай 2: $n \neq o$}
\paragraph*{}
Нека $n = s(x)$. Тогава $n+m = s(x) + m = s(x+m)$ от дефиницията на събирането. От друга страна $m+n = m + s(x) = s(m+x) = s(x+m)$ от подзадача 1 и от индукционното предположение. Така доказахме случая. 

\end{document}