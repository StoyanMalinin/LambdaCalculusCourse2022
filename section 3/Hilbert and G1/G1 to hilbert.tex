\documentclass[12pt]{article}
\usepackage{lingmacros}
\usepackage{tree-dvips}
\usepackage[utf8]{inputenc}
\usepackage[russian]{babel}
\usepackage{amsmath,amssymb}
\usepackage{multirow}
\usepackage{hyperref}
\usepackage{caption}
\usepackage{tabularx}
\usepackage{framed,enumitem} 


\begin{document}

\newcommand\alphaeq{\mathrel{\stackrel{\makebox[0pt]{\mbox{\normalfont\tiny $\alpha$}}}{=}}}
\newcommand\betaeq{\mathrel{\stackrel{\makebox[0pt]{\mbox{\normalfont\tiny $\beta$}}}{=}}}
\newcommand\betaetaeq{\mathrel{\stackrel{\makebox[0pt]{\mbox{\normalfont\tiny $\beta\eta$}}}{=}}}

\paragraph*{}
Целта е да докажем, че за всяко доказателство от G1 има доказателство за същото твърдение със същите допускания и в H. Тоест, ако $\Gamma \implies \Delta$ в G1 и $\Delta \neq \{ \}$, то ще има поне едно доказателство в Н, което извежда нещо от $\Delta$.

\paragraph*{}
Ще решим задачата като направим пълна индукция по дълбочината на дървото. 

\subsection*{Случай $Ax$}
\paragraph*{}
\begin{enumerate}
    \item A (As)
\end{enumerate}

\subsection*{Случай $L \bot$}
Случаят не се разглежда, понеже $\Delta = \{\}$

\subsection*{Случай $LW$}
Използваме доказателството на $\Gamma \implies \Delta$ и "игнорираме" $\;$ допускането $A$. 

\subsection*{Случай $RW$}
Използваме доказателството за $\Gamma \implies \Delta$.

\subsection*{Случай $LC$}
Тъй като при Хилбертовите системи допусканията са множество, то няма разлика между $A, A, \Gamma$ и $A, \Gamma$ и по тази причина можем да използваме докзателството на $A, \Gamma \implies \Delta$.

\subsection*{Случай $RC$}
Използваме доказателството за $A, \Gamma \implies \Delta, A, A$.

\subsection*{Случай $L \land$}
Генерираме следното доказателство при допускания $\Gamma A_0 \land A_1$
\begin{enumerate}
    \item $A_0 \land A_1$ (As)
    \item $(A_0 \land A_1) \rightarrow A_i$ (Ax)
    \item $A_i$ (MP 1, 2) 
\end{enumerate}

След това залепяме доказателството на $\Gamma, A_i$ отзад като махаме всички случаи, в които $A_i$ се използва като (As).

\subsection*{Случай $R \land$}
Нека имаме $X_1, X_2, ..., X_n$ е доказателство на $\Gamma \implies \Delta, A$ и $Y_1, Y_2, ..., Y_m$ е доказателство на $\Gamma \implies \Delta, B$. Ако $X_n \in \Delta$ или $Y_m \in \Delta$, тогава използваме съотвентото доказателство. В противен случай, знаем, че $X_n = A$ и $Y_m = B$. Тогава, залепяме двете доказателства и добавяме следната последователност:
\begin{itemize}[itemindent=4em]
    \item[n+m+1: ] $A \rightarrow B \rightarrow A \land B$ (Ax)    
    \item[n+m+2: ] $B \rightarrow A \land B$ (MP n, n+m+1), тук на позиция $n$ седи $X_n$ 
    \item[n+m+3: ] $A \land B$ (MP n+m, n+m+2), тук на позиция $n+m$ седи $Y_m$.
\end{itemize}

\subsection*{Случай $L \lor$}
Нека $O \in \Gamma$. По предположение знаем, че съществуват доказателства на $A, \Gamma \implies \Delta$ и $B, \Gamma \implies \Delta$. Нека те да бъдат съотвенто $X_1, X_2, ..., X_n$ и $Y_1, Y_2, ..., Y_m$, където $X_n, Y_m \in \Delta$. От теорема за дедукцията знаем, че съществуват доказателства и на $\Gamma \implies A \rightarrow O$ и $\Gamma \implies B \rightarrow O$. Нека тези доказателства са $U_1, U_2, ..., U_k = A \rightarrow X_n$ и $V_1, V_2, ..., V_l = B \rightarrow Y_m$. Залепяме двете доказателства едно след друго и към тях слагаме следната последователност:
\begin{itemize}[itemindent=4em]
    \item[k+l+1: ] $ (A \rightarrow X_n) \rightarrow (B \rightarrow Y_m) \rightarrow A \lor B \rightarrow O $ (Ax)    
    \item[k+l+2: ] $(B \rightarrow Y_m) \rightarrow A \lor B \rightarrow O$ (MP k, k+l+1), тук на позиция $n$ седи $U_k$ 
    \item[k+l+3: ] $A \lor B \rightarrow O$ (MP k+l, k+l+2), тук на позиция $k+l$ седи $V_l$ 
    \item[k+l+4: ] $ A \lor B $ (As) 
    \item[k+l+5: ] $O$ (MP k+l+4, k+l+3)
\end{itemize}

\section*{Случай $R \lor$}
Нека имаме доказателство $X_1, X_2, ..., X_n$ на $\Gamma \implies \Delta, A_i$. Ако $X_n \in \Delta$, то просто използваме това доказателство. В противен случай, знаем, че $X_n = A_i$. Тогава към това доказателство добавяме $X_{n+1} = A_i \rightarrow A_0 \lor A_1 \; \; (Ax)$ и $X_{n+2} = A_0 \lor A_1$ MP(n, n+1). 

\section*{Случай $L \rightarrow$}
ОТ ИП имаме доказателства $X_1, X_2, ..., X_n$ и $Y_1, Y_2, ..., Y_m$ за $\Gamma \implies \Delta, A$ и $B, \Gamma \implies \Delta$. От теорема на дедукцията знаем, че имаме доказалтество $Z_1, Z_2, ..., Z_k$ на $\Gamma \implies B \rightarrow Y_m$. Ако $X_n \in \Delta$, тогава можем да използваме $X_1, X_2, ..., X_n$ като доказалтество на $A \rightarrow B, \Gamma \implies \Delta$. В противен случай, знаем, че $X_n = A$ и правим следната конструкция.  

\paragraph*{}
Залепяме доказателствата $X_1, X_2, ..., X_n$ и $Z_1, Z_2, ..., Z_k$ едно след друго и добавяме следната последователност:
\begin{itemize}[itemindent=4em]
    \item[n+k+1: ] $A \rightarrow B$ (As)
    \item[n+k+2: ] $B$ MP(n, n+k+1), тук на позиция $n$ стои $X_n$  
    \item[n+k+3: ] $Y_m$ MP(n+k+2, n+k), тук на позиция $n+k$ стои $Z_k = B \rightarrow Y_m$  
\end{itemize}

\section*{Случай $R \rightarrow$}
Следва директно от теоремата за дедукцията.

\section*{Случай $L \forall$}
Нека $X_1, X_2, ..., X_n$ е доказалтество на $\Gamma, A[x \longmapsto t] \implies \Gamma$. Слагаме $C_1 = \forall_x A \rightarrow A[x \longmapsto t]$ (Ax), $C_2 = \forall_x A$ (As), $C_3 = A[x \longmapsto t]$ MP(1, 2). Сега залепяме $X_1, X_2, ..., X_n$ като вече не използваме $A[x \longmapsto t]$ като допускане, а като резултат.

\section*{Случай $R \forall$}
Знаем, че имаме доказателство $X_1, X_2, ..., X_n$ на $\Gamma \implies \Delta, A$. Ако $X_n \in \Delta$ можем да използваме това доказателство и за $\Gamma \implies \Delta, \forall_x A$, в противен случай знаем, че $X_n = A$. Тогава можем да довършим като използваме теорема за генерализацията.

\section*{Случай $L \exists$}
Знаем, че имаме доказателство $X_1, X_2, ..., X_n \in \Delta$ на $A, \Gamma \implies \Delta$. От теоремата за дедукцията знаем, че съществува доказателство $Z_1, Z_2, ..., Z_k = A \rightarrow X_n$ за $\Gamma \implies A \rightarrow X_n$. Тъй като $x \not\in FV(\exists_x A, \Gamma)$, то от теоремата за генерализацията знаем, че съществува доказателство $Y_1, Y_2, ..., Y_m = \forall_x (A \rightarrow X_n)$ за $\Gamma, \exists_x A \implies \forall_x (A \rightarrow \exists_x A)$. Правим доказателството $Z_1, ..., Z_k, Y_1, ..., Y_m$ и към него добавяме:
\begin{itemize}[itemindent=4em]
    \item[k+m+1: ] $\forall_x(A \rightarrow X_n) \rightarrow (\exists_x A \rightarrow X_n)$ (Ax).
    \item[k+m+2: ] $\exists_x A \rightarrow X_n$  MP(k, n+m+2), тук на позиция $k$ седи $\forall_x(A \rightarrow X_n)$.
    \item[k+m+3: ] $\exists_x A$ (As)
    \item[k+m+4: ] $X_n$ MP(k+m+3, k+m+2)    
\end{itemize}
Тъй като $X_n \in \Delta$, то това докзателство върши работа.

\section*{Случай $R \exists$}
Нека имаме доказателството $X_1, X_2, ..., X_n$ на $\Gamma \implies \Delta, A[x \longmapsto t]$. Ако $X_n \in \Delta$, то можем да използваме това доказателство. В противен случай, знаем, че $X_n = A[x \longmapsto t]$. Тогава към това доказателство залепяме следната последователност:
\begin{itemize}[itemindent=4em]
    \item[n+1: ] $A[x \longmapsto t] \rightarrow \exists_x A$ (Ax)
    \item[n+2: ] $\exists_x A$ MP(n, n+1), тук на позиция $n$ седи $X_n = A[x \longmapsto t]$. 
\end{itemize}

\end{document}